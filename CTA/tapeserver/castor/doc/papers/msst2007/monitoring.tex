Monitoring is an important component of CASTOR. It is used
by the operations team, by the user community, and to aid in
planning future resource allocation. This section presents a
brief overview of the monitoring system, a detailed explanation is
available in \cite{chepmon}.

The CASTOR monitoring system queries the state of the system, in terms
of performance and errors, in a variety of ways. We make use of DLF
to centrally analyze logs from all machines within the
system. SQL procedures are used on each instance database to provide
status information to a variety of presentations systems.

In addition to the CASTOR specific software, we make use of Computer
Centre monitoring tools such as Lemon \cite{lemon}. For example, we
make  use of \textit{Lemon actuators} which act as triggers for
specific monitoring metrics. When specific conditions are met, actions
are automatically  taken. These can range from restarting a daemon, to
moving a disk server out of production, to sending a GSM SMS to a
service manager.

The monitoring of resource utilization provides accounting information
that is used for planning the allocation of resources. We
record information about the number of requests per second to the
stager, the number of files in the system and their status, disk and
tape space usage and a cornucopia of other information.

To make ensure system consistency, checking mechanisms are used
for configuration and for data consistency. The
{\bf castorReconcile} package has been developed to present  information
from the CASTOR database, the scheduling subsystem, the state
management system, and others. It provides  a summary of each instance
so that all information about a problem machine can be seen in a
single place, and the state of all machines in an instance can be seen
at a glance. In addition, we have software which periodically scans
all the file systems and compare the file metadata on the disk servers
with the status described by the stager database to detect
deviations. This detects problems, such as orphaned files, missing
disk copies, size mismatches and misplaced files. It is planned to add
checksums to this to verify the integrity of files.

\subsection{Monitoring Evolution}

Initially CASTOR monitoring was fully integrated with the monitoring
system used at CERN. However, to aid deployment at other sites, since
2006 we have made the software more independent of the monitoring tool
used for aggregation, transport and display of the information.  To
achieve this we run as much monitoring as possible at the database
layer, where SQL procedures aggregate information and make it
available to various display systems.  This feature has increased the
monitoring systems flexibility and  efficiency, and has also resulted
in our sensors becoming  predominantly thin clients. These thin
clients can easily be  developed for other monitoring systems.

