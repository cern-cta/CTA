
The CASTOR system supports a number of protocols by means of \textit{movers}.
Movers implement the data transfer protocols used
to access data on the diskservers as well as on the tertiary storage.
We classify movers as internal and external. Internal movers are integrated 
with CASTOR and used as native protocols. In addition, they are able to deal with CASTOR TURLs, 
i.e. TURLs which refer to files in the CASTOR namespace (for instance,
\texttt{protocol://diskserver:port//castor/cern.ch/...}). The protocols included in
this category are RFIO, ROOT, XROOT, and GridFTPv2. On the other hand, external 
movers contact CASTOR exclusively via the public client API, and they 
don't handle CASTOR TURLs. GridFTPv1 is included in this category.\\

RFIO is the original native CASTOR protocol, and it provides a POSIX compliant file I/O API interface,
which includes functions like \textit{rfio\_open(), rfio\_read(), rfio\_lseek(), rfio\_stat()}, etc.
RFIO, ROOT and GridFTP v2 are scheduled to start on a disk server for each file access. The I/O transfer with
the client is handled by the mover, which is spawned on demand to connect to the client according
to the requested protocol. The mover is authorized to deal with the scheduled request only,
preventing clients from misbehaviours against the disk cache content.

XROOT is able to handle concurrent file accesses using a single access
from the CASTOR point of view. It also keeps a catalogue with
the physical file location, so that future accesses to the same file won't need
another operation on the CASTOR system. Moreover, while RFIO and ROOT are
provided by all diskservers, XROOT is typically deployed on a dedicated disk pool,
so that a double disk cache level can be implemented.\\

GridFTP v1 is an external mover and GridFTP v2 can behave as
an external mover as well. They run on the disk servers independently from CASTOR,
and they use RFIO as an internal mover to access the data and serve the GridFTP clients.
