
\subsection{Tape Archive Management}

The CASTOR tape archive provides long term availability
of data with maximum transparency and flexibility for the end user.
In other words, users should not realize that their data are
stored on tape and the data should be automatically migrated to
newer media when necessary without any impact on their availability.
CASTOR's tape system addresses these issues via a set
of components.

The {\bf tape daemon} is the interface to the tape hardware. It interacts
with the tape robots to transfers data to/from tape.
The tape daemon supports a number of types of tape drives and related robots.
The current status of CERN's tape robots and drives is described in
Table~\ref{tapehardware}.

\begin{table}[htbp]
  \begin{center}
  \begin{tabular}{|c|c|c|}
    \hline
      Drive type     & Nb drives & Speed \\
    \hline
      STK 9940B      & 44        & 30 MB/s \\
      IBM 3592 E05   & 50        & 100 MB/s \\
      STK T10000     & 50        & 120 MB/s \\
    \hline
    \hline
      Library        & Nb tapes & Capacity \\
    \hline
      STK Powderhorn & 10,000   & 2 PB \\
      IBM 3584       &  5,500   & 3.85 PB \\
      STK SL8500     & 13,000   & 6.5 PB \\
    \hline
  \end{tabular}
  \end{center}
  \caption{Status of the tape hardware at CERN}
  \label{tapehardware}
\end{table}

The {\bf Volume Manager} (VMGR) and the Volume and Drive Queue Manager (VDQM)
respectively handle the status of the tapes and tape drives. They handle
queues of requests from tapes and drives and act as resource allocators.

The {\bf remote tape copy} (rtcopy) software interacts with the disk layer.
It manages a set of migrators and recallers respectively writing and reading
data from/to tape. Both migrations and recalls are user driven actions
that depend on the disk cache status and are controlled by external policies.
They use high speed streams and allow for multiplexing of streams from
different disk servers in order to fill the tape buffers.
rtcopy also takes care of computing a checksum of the files going to, and
coming from, tape in order to detect possible tape errors.

Finally, the {\bf repack} component is responsible for moving data from a set of
tapes to another set. This allows both to migrate data from one generation
of tapes to another and to recuperate space on tapes where a lot of files
were deleted.
