\chapter{When things go wrong...}
When something goes wrong while using the CASTOR commands or libraries please check the CASTOR FAQ
\url{http://cern.ch/castor/faq.htm} if it is a known problem. If not, can contact
the local CASTOR support (for CERN users, {\bf Castor.Support@cern.ch}). In this section you
will find some instruction to help speeding up the problem resolution.

\section{Provide as much as possible all relevant information from the beginning}
For an efficient and rapid problem diagnosis, it is important to provide a maximum
of relevant information, which should include at least the following:
\begin{itemize}
   \item The client host name where you tried the CASTOR command. Use the
        {\tt hostname} or {\tt uname} commands
   \item The approximate date and time for the problem. Use the {\tt date}
        command
   \item Your complete environment setting: use the {\tt env} command to list
        your environment
   \item The full command line and error text of the failing command
   \item The CASTOR release installed on the host where you tried the CASTOR command. Attach
         the output of the {\tt rpm -qa |grep -i castor} or {\tt castor -v} commands to your
         problem report
   \item The path of the command that failed: {\tt which <command-name>},
         e.g. {\tt which rfcp}
   \item The shared library dependency: {\tt ldd <command-name>}, e.g. {\tt ldd rfcp}
\end{itemize}

Example (problem with the rfcp command):
\small
\begin{verbatim}
[lxplus] uname -a >castorProblemReport.txt
[lxplus] date >>castorProblemReport.txt
[lxplus] env >>castorProblemReport.txt
[lxplus] rpm -qa| grep castor >>castorProblemReport.txt
[lxplus] castor -v >>castorProblemReport.txt
[lxplus] which rfcp >>castorProblemReport.txt
[lxplus] ldd `which rfcp` >>castorProblemReport.txt
\end{verbatim}
\normalsize

\section{CASTOR traces}
When using RFIO, ROOT or CASTOR stager client commands, it is possible to also get an exhaustive
debug trace by setting the environment variables {\tt STAGER\_TRACE=3} and {\tt RFIO\_TRACE=3}. Example:
\small
\begin{verbatim}
[lxplus] rfcp newTrigger.root /castor/cern.ch/user/l/linda/higgs/newTrigger.root
/castor/cern.ch/user/l/linda/higgs/newTrigger.root : Entry not found
[lxplus] setenv RFIO_TRACE 3
[lxplus] setenv STAGER_TRACE 3
[lxplus] rfcp newTrigger.root /castor/cern.ch/user/l/linda/higgs/newTrigger.root
    **** : trace level set to 3
  rfio: rfio_stat64(/castor/cern.ch/user/l/linda/higgs/newTrigger.root, bfffd380)
    rfio: rfioTURLFromString does not recognize TURL
    rfio: rfio_parseln() check /castor/cern.ch/user/l/linda/higgs/newTrigger.root§ against castor root /castor
    rfio: rfio_parseln() call Cns_selectsrvr(/castor/cern.ch/user/l/linda/higgs/newTrigger.root,0x9ac4618,0x9ac4200,0xbfff51d8)
    rfio: rfio_parseln() Cns_selectsrvr() returns host=cnsuser.cern.ch, path=/castor/cern.ch/user/l/linda/higgs/newTrigger.root
  rfio: rfio_stat64: /castor/cern.ch/user/l/linda/higgs/newTrigger.root is an HSM path
   rfio: rfioTURLFromString does not recognize TURL
   rfio: rfio_parseln() check /etc/group against castor root /castor
   rfio: rfio_parseln() Using site-wide NFS ROOT "/shift"
   rfio: rfioTURLFromString does not recognize TURL
   rfio: rfio_parseln() check /castor/cern.ch/user/l/linda/higgs/newTrigger.root against castor root /castor
   rfio: rfio_parseln() call Cns_selectsrvr(/castor/cern.ch/user/l/linda/higgs/newTrigger.root,0x9ac4618,0x9ac4200,0xbfff5338)
   rfio: rfio_parseln() Cns_selectsrvr() returns host=cnsuser.cern.ch, path=/castor/cern.ch/user/l/linda/higgs/newTrigger.root
  rfio: rfio_stat64(newTrigger.root, bfffd3e0)
    rfio: rfioTURLFromString does not recognize TURL
    rfio: rfio_parseln() check newTrigger.root against castor root /castor
    rfio: rfio_parseln() Using site-wide NFS ROOT "/shift"
  rfio: rfio_stat64: using local stat64(newTrigger.root, bfffd3e0)
 rfio: rfcp: Setting stager log callback
  rfio: rfio_open64_ext(newTrigger.root, 00, 0644, 0, 0, 0, )
    rfio: rfioTURLFromString does not recognize TURL
    rfio: rfio_parseln() check newTrigger.root against castor root /castor
    rfio: rfio_parseln() Using site-wide NFS ROOT "/shift"
  getifnam_r: getifnam_r(3) entered
   getifnam_r: getsockname returned 88
  rfio: rfio_open64_ext(/castor/cern.ch/user/l/linda/higgs/newTrigger.root, 01101, 0644, 0, 0, 0, )
    rfio: rfioTURLFromString does not recognize TURL
    rfio: rfio_parseln() check /castor/cern.ch/user/l/linda/higgs/newTrigger.root against castor root /castor
    rfio: rfio_parseln() call Cns_selectsrvr(/castor/cern.ch/user/l/linda/higgs/newTrigger.root,0x9ac4618,0x9ac4200,0xbfff5160)
    rfio: rfio_parseln() Cns_selectsrvr() returns host=cnsuser.cern.ch, path=/castor/cern.ch/user/l/linda/higgs/newTrigger.root
  rfio: rfio_open64_ext: /castor/cern.ch/user/l/linda/higgs/newTrigger.root is an HSM path
   rfio: Calling stage_open with: /castor/cern.ch/user/l/linda/higgs/newTrigger.root 241 1a4
    **** : trace level set to 3
    stager: stage_put Usertag=NULL Protocol=rfio File=/castor/cern.ch/user/l/linda/higgs/newTrigger.root mode=420/size=0
    stager: Opt SVCCLASS=zzzz
    stager: Setting euid: 12345
    stager: Setting egid: 1234 
    stager: Creating socket for stager callback
    stager: Will wait for stager callback on port 39360
    stager: Aug 24 22:13:19 (1156450399) Sending request
    stager: Waiting for acknowledgement
    stager: 44ee085f-0000-1000-a75f-e37f230a0000 SND 0.02 s to send the request
    stager: Request sent to RH - Request ID: 44ee085f-0000-1000-a75f-e37f230a0000
    stager: Waiting for callback from stager
    stager: 44ee085f-0000-1000-a75f-e37f230a0000 CBK 0.02 s before callback was received
   rfio: stage_open error: 1014/No such service class "zzzz"
   rfio: rfio_serror: errno=0, serrno=1014, rfio_errno=0
/castor/cern.ch/user/l/linda/higgs/newTrigger.root : Entry not found
[lxplus]
\end{verbatim} 
\normalsize

The problem in the above example is a wrong setting of the service class (disk pool).

Always provide the trace output with your problem report if possible. However, as can be seen
above the trace is rather long and if the error happens after a long execution, it may not always
be possible to attach the trace. In that case, save it to a file in your AFS public directory
and provide the filename in your problem report.

\section{Hanging transfers}
When you launch an application or CASTOR command (e.g. rfcp) and it appears to hang, it can
be stuck for various reasons. The most frequent causes are:
\begin{itemize}
  \item{\bf Waiting on tape recall - } The file is not available on disk and a recall from tape is
                                necessary. Check with {\tt stager\_qry -M ....} to see if
                                that is the case and, if so, the request would normally appear
                                in the tape queue. See Section \ref{sect:tapepbs} for more
                                details how to check the tape queue and status.
                                If no tape request is found it can sometimes help clearing 
                                the file with {\tt stager\_rm -M ...} and retry
                                the request with {\tt stager\_get -M ...}.
  \item{\bf Local firewall - } If you run your CASTOR client on a desktop computer it may be that
                           the transfer hangs because the CASTOR call-back is blocked by the
                           local firewall (iptables). CASTOR normally needs inbound access in the port
                           range [30,000 - 30,100] but a different range can be configured in {\tt /etc/castor/castor.conf}, for instance:
  \begin{itemize}
    \item CLIENT LOWPORT 50000 
    \item CLIENT HIGHPORT 50100
  \end{itemize}
  \item{\bf Waiting on transfer slot - } If the file resides on disk server that has no more available
                                transfer slots, new requests for accessing files on that server
                                may be queued up. For the moment there is no client command to
                                check if this is the case.
  \item{\bf Slow data transfer - } The data transfer may actually be ongoing but progressing very
                            slowly due to network bandwidth limitations. If it is a transfer
                            into the client machine, you may monitor the timestamp and size of
                            target file.
\end{itemize}
