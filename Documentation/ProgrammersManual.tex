\chapter*{Preface}
\addcontentsline{toc}{chapter}{Preface}

The Tape server project is targeted at replacing the CASTOR tape server with a new drop-{}in reimplementation. The reimplementation will replace a legacy implementation that is written in C.

% ------- 
% Chapter 
% ------- 

\chapter{Developer's manual}

\section{Targeted environment}

CERN SLC5 and SLC6, 64bits. Although it should compile in theory, the 32 bits version is not tested. [ TODO: prevent 32 bits compilation ]

\section{Reference documentations}
\subsection{SCSI specifications}

The SCSI commands can be found in the SCSI section of hackipedia:
 \href{http://hackipedia.org/Hardware/SCSI/}{http://hackipedia.org/Hardware/SCSI/}   The most significant sections for tape server development are the stream commands
 \href{http://hackipedia.org/Hardware/SCSI/Stream%20Commands/SCSI%20Stream%20Commands%20-%203.pdf}
    {http://hackipedia.org/Hardware/SCSI/Stream Commands/SCSI Stream Commands - 3.pdf} 
 and the SCSI primary commands
    \href{http://hackipedia.org/Hardware/SCSI/Primary%20Commands/SCSI%20Primary%20Commands%20-%204.pdf}
      {http://hackipedia.org/Hardware/SCSI/Primary Commands/SCSI Primary Commands - 4.pdf}.

\subsection{SCSI support in Linux}
On the Linux side, the main references are the Linux 2.4 SCSI subsystem HOWTO
\href{http://mirrors.kernel.org/LDP/HOWTO/pdf/SCSI-2.4-HOWTO.pdf}{http://mirrors.kernel.org/LDP/HOWTO/pdf/SCSI-2.4-HOWTO.pdf},
especially for is section 9.3 on the st driver,
and the Linux SCSI Generic (sg) HOWTO 
 \href{http://mirrors.kernel.org/LDP/HOWTO/pdf/SCSI-Generic-HOWTO.pdf}{http://mirrors.kernel.org/LDP/HOWTO/pdf/SCSI-Generic-HOWTO.pdf}.

More details regarding the Generic SCSI driver can be found in this web site: 
\href{http://sg.danny.cz/sg/}{http://sg.danny.cz/sg/}.

The section on the SG\_IO ioctl, \href{sg.danny.cz/sg/sg\_io.html}{sg.danny.cz/sg/sg\_io.html} details the usage of the 
simples ioctl for the generic SCSI driver, which allows the invocation of a SCSI command and the collection of the 
result in a single system call.

\subsection{Unsorted CASTOR docs}
There is a collection of links to various documentations written in the past on this page:
\href{http://castorwww.web.cern.ch/castorwww/links.htm}{http://castorwww.web.cern.ch/castorwww/links.htm}.

\section{Tools used during development}
\subsection{ Required tools for build}
\begin{itemize}
\item{}mhvtl (\href{https://sites.google.com/site/linuxvtl2/}{https://sites.google.com/site/linuxvtl2/}) for developing against virtual drives and libraries.
\item{}GCC/G++ (Basic SLC version)
\item{}CMake (Basic SLC version)
\item{}rpmbuild (Basic SLC version)
\item{}Google Mock/Google test (GTest is provided in EPEL repository for SLC. 
  GMock requires recompilation (TODO: link to source)
\item{}Valgrind (Basic SLC version)
\item{}\LaTeX (Basic SLC version)
\item{}Doxygen for code documentation (Basic SLC version)
\end{itemize}

\subsection{Tools used during development}
\begin{itemize}
\item{}TeamCity for continuous integration
\item{}NetBeans as an IDE, including for remote development\
\end{itemize}

\section{Classes layout}

\begin{lstlisting}[caption=Code example,label=code1]
namespace Test {
  class Test {
  public:
    Test() {}
  };
} // namespace Test
\end{lstlisting}


