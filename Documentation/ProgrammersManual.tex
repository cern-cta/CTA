\chapter*{Preface}

The Tape server project is targeted at replacing the CASTOR tape server with a new drop-{}in reimplementation. The reimplementation will replace a legacy implementation that is written in C.

% ------- 
% Chapter 
% ------- 

\chapter{Developer's manual}

\section{Targeted environment}

CERN SLC5 and SLC6, 64bits. Although it should compile in theory, the 32 bits version is not tested. [ TODO: prevent 32 bits compilation ]

\section{Reference documentations}

The SCSI commands can be found in the SCSI section of hackipedia:
 \href{http://hackipedia.org/Hardware/SCSI/}{http://hackipedia.org/Hardware/SCSI/}. 
The most significant sections for tape server development are the stream commands
 \href{http://hackipedia.org/Hardware/SCSI/Stream%20Commands/SCSI%20Stream%20Commands%20-%203.pdf}{
    http://hackipedia.org/Hardware/SCSI/Stream Commands/SCSI Stream Commands - 3.pdf} and the 
\href{http://hackipedia.org/Hardware/SCSI/Primary%20Commands/SCSI%20Primary%20Commands%20-%204.pdf}{SCSI primary commands}.

On the Linux side, the main references are the Linux SCSI Generic (sg) HOWTO 
 \href{http://tldp.org/HOWTO/SCSI-Generic-HOWTO/index.html}{  http://tldp.org/\-HOWTO/\-SCSI-{}Generic-{}HOWTO/\-index.html  }    , and ... TBC.

http://sg.danny.cz/sg/sg\_io.html

http://sg.danny.cz/sg/index.html

\section{Tools used during development}

\subsection{ Required tools for build}
\begin{itemize}

\item{}GCC/G++ (Basic SLC version)


\item{}CMake (Basic SLC version)


\item{}rpmbuild (Basic SLC version)

\end{itemize}

Google Mock/Google test (GTest is provided in EPEL repository for SLC. GMock requires recompilation (TODO: link to source)

Valgrind (Basic SLC version)

Docbook (Basic SLC version)

Doxygen for code documentation

\subsection{Tools used during development}
\begin{itemize}

\item{}TeamCity for continuous integration


\item{}NetBeans as an IDE, including for remote development

Syntext Serna Free for docbook documents edition

\end{itemize}

\end{document}
