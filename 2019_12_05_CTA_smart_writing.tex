\documentclass[aspectratio=149]{beamer}
\usepackage[utf8]{inputenc}
\usepackage[super]{nth}
\usepackage{censor}
\usetheme{cern}

% ADC meetings: Data carousel | (discussion) 
% Thursday, 5 December   40/S2-D01 - Salle Dirac
% https://indico.cern.ch/event/823341/#day-2019-12-05


%09:00 → 10:30
%ADC meetings: Data carousel | (discussion)¶ 40/S2-D01 - Salle Dirac
%Conveners: Alexei Klimentov (Brookhaven National Laboratory (US)), David Cameron (University of Oslo (NO)), David Michael South (Deutsches Elektronen-Synchrotron (DE)), Johannes Elmsheuser (Brookhaven National Laboratory (US)), Mario Lassnig (CERN), Xin Zhao (Brookhaven National Laboratory (US))
%
%    09:00
%    Materials : Tier-1s presentations (October-November splinter meetings), FTS, Rucio, ProdSys2¶ 1m
%
%09:01
%Meeting goals¶ 10m
%
%can we agree on the way forward to cut the following two gaps in the meeting
%
%Gap 1 : between the throughput out of the tape system itself and the throughput delivered to users (rucio in this case)
%--This is to tackle the issues along the staging chain, dCache, FTS, Rucio, PS2, etc, to minimize performance penalties to the original tape throughput
%Gap 2 : between the nominal tape throughput and the current throughput out of tape
%--This is about “smart writing”, by bigger files and/or better organizing files among tapes, to reach higher tape reading efficiency
%Speakers: Alexei Klimentov (Brookhaven National Laboratory (US)), Mario Lassnig (CERN), Xin Zhao (Brookhaven National Laboratory (US))
%09:11
%Summary and highlights of Oct-Nov discussions with Tier 1s and CERN¶ 20m
%Speakers: Alexei Klimentov (Brookhaven National Laboratory (US)), Mario Lassnig (CERN), Xin Zhao (Brookhaven National Laboratory (US))
%09:31
%ATLAS Sites (topical discussion)¶ 59m
%
%    Planning / notification / feedback /discussions
%    What concerns sites have with tape system throughput (writing/reading) ?
%    -- Active tape drives usage vs drives lifetime (data staging scenario)
%    -- What are the bottlenecks at sites, any planned improvements ?
%    -- What help do they need from ADC, dCache, FTS ? (e.g. sites want the staged files to be transferred out of disk buffer as soon as possible, avoid too many staging retries ...)
%    Site staging profile
%    -- Possible extension ?
%    What can sites do w.r.t. “Smart writing” ?
%    -- new features in tape system ?
%    -- external users (eg. ADC/dCache) to create bigger files and/or better organize the order of files when writing to tape ?
%    What are the special concerns for sites supporting multiple-VO ?
%    -- Do we understand requirements and intended usage of tape resources of other VOs ?
%    monitoring tools enough for sites? any suggestions on improving monitoring ?
%
%11:00 → 12:30
%ADC meetings: Data carousel (II)¶ 40/S2-D01 - Salle Dirac
%Conveners: Alexei Klimentov (Brookhaven National Laboratory (US)), David Cameron (University of Oslo (NO)), David Michael South (Deutsches Elektronen-Synchrotron (DE)), Johannes Elmsheuser (Brookhaven National Laboratory (US)), Mario Lassnig (CERN), Xin Zhao (Brookhaven National Laboratory (US))
%
%    11:00
%    dCache global view and development roadmap¶ 25m
%    Speaker: Mr Tigran Mkrtchyan (DESY)
%    11:25
%    Topical discussion¶ 1h
%        7 out of 10 T1s run dCache as tape frontend
%        Can dCache provide common solution to improve scalability of dCache HSM interface, something like ENDIT does for TSM sites, that will benefit all dCache sites ?
%        Can dCache group files by datasets in writing to tapes, which will benefit all dCache T1s ?
%        other tape frontend (CTA, StoRM)
%        -- plans ?
%        alternative protocol for tape access ?
%        -- SRM vs QoS APIs ?

 

% Based on CERN beamer theme by Jerome Belleman https://github.com/jeromebelleman/beamer-cern

% The optional `\author` command defines the author and is displayed in the slide produced by the `\titlepage` command.
\author{Eric Cano on behalf of the CTA team}

% The optional `\title` command defines the title and is displayed in the slide produced by the `\titlepage` command.
\title{CTA Smart writing}

% The optional `\subtitle` command will add a smaller title below the main one, and will not be displayed in any of the slides' footer.
\subtitle{Conceptual proposition}

% The optional `\date` command will display a custom free text date on the all of the slides' footer. If omitted today's date will be used.
\date{ADC meetings: Data carousel, 5 Dec 2019}

\setcounter{tocdepth}{1}

\begin{document}

\frontcover

% The optional `\titlepage` command will create a slide with the presentation's title, subtitle and author.
\frame{\titlepage}

% The optional `\tableofcontents` command will automatically create a table of contents based pm the sections.
\frame{\tableofcontents}

\section{The problem}
\begin{frame}{What are we trying to solve?}
  \begin{itemize}
    \item Datasets are always read whole
    \item Tape systems not dataset-aware during write
    \begin{itemize}    
      \item Files scattered over tapes $\Rightarrow$ more read mounts
      \item Files interleaved with others within tape
      
      $\Rightarrow$ drive spends time positioning on reads
    \end{itemize}
  \end{itemize}
\end{frame}

\section{Files tagging by dataset name}
\begin{frame}{Files tagging by dataset name}
\begin{itemize}
  \item Per-file property
  \item Type = string
  \begin{itemize}
    \item Can we define a length cap? 
  \end{itemize}
  \item Only rely on comparison (no ordering, ranking...)
\end{itemize}
\end{frame}

\section{User interface}
\begin{frame}{File tagging}
\begin{itemize}
  \item On write, per file tagging
  \begin{itemize}
    \item Has to go through Rucio/FTS/EOS/CTA
  \end{itemize}
  \item Back tagging of existing files (several scenarios)
  \begin{itemize}
    \item Executed as a one-off, we could have rule based update script
    \item More general: provide get/set operation per file and leave it to the user
  \end{itemize}
\end{itemize}
\end{frame}

\section{Tape system optimization}
\begin{frame}{Tape system optimizations}
\begin{itemize}
  \item Write optimization
  \begin{itemize}
    \item Divide archive queue in per-dataset sub-queue
    \item Make write mounts stick to a dataset (until it is drained)
    \begin{itemize}
      \item $\Rightarrow$ Contiguous files, zero positioning on read
    \end{itemize}
    \item Possibly cap the per-dataset parallel writes
    \begin{itemize}
      \item $\Rightarrow$ Soft-limiting the spreading over tapes
    \end{itemize}
  \end{itemize}
  \item Repack/defrag
  \begin{itemize}
    \item Repack can then write in an optimized manner (defrag)
    \item Repack input (which files to read) could be dataset driven instead of tape driven
    \begin{itemize}
      \item If extra read mount cost bearable
      \item Will have to take into account tape level constraints as well (will it be worth the complexity?)
    \end{itemize}
  \end{itemize}
\end{itemize}
\end{frame}

\section{Possible bonus features}
\begin{frame}{Possible bonus features}
\begin{itemize}
  \item Multi-level tagging, allowing to better choose the {\em{}next} dataset in a mount
  \item Retrieve by dataset (implies big changes in whole data transfer chain, and possibly hairy error handling)
\end{itemize}
\end{frame}
%	a.	Tagging of new files, as they are written
%	b.	Back tagging of pre-existing files
%	… and of course user should be able to query this tag.
%	2)	Queueing by tag in archive queue. This implies some changes in data structure (having sub-queue in archive queues). Once this is done, it is trivial to make the archive session sticky and make sure it will not switch to a second dataset.
%	
%	A bonus feature could be to add a second/several layer(s) of tagging (dataset of dataset) which would allow orienting the choice of the next dataset in a mount after finishing with one, but this is really secondary optimisation (but not expensive either while we’re at it).
%	
%	Once the old files are tagged, we could repack by dataset(s) to defragment the existing data (instead of tape oriented repack).

\backcover

\end{document}
