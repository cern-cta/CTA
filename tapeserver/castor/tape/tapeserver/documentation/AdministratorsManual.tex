% ------- 
% Chapter 
% ------- 

\chapter{Administrator's manual}
\section{User and capabilities}
Castor used to run as root, which is not the best safety policy. New version should be run by stage:st (even if still own by root),
because the first action will be to drop the root's privlegeves to move to stage:st
But accessing /dev/nst* for writing data requires to either be root (on SLC5) or to have the the capabilitie CAP\_SYS\_RAWIO set on. 
Here the incriminated piece of code into the st driver :
\begin{table}[h]
\begin{lstlisting}
       switch (cmd_in) {
              case SCSI_IOCTL_GET_IDLUN:
              case SCSI_IOCTL_GET_BUS_NUMBER:
                     break;
              default:
                     if ((cmd_in == SG_IO ||
                          cmd_in == SCSI_IOCTL_SEND_COMMAND ||
                          cmd_in == CDROM_SEND_PACKET) &&
                         !capable(CAP_SYS_RAWIO))
                           i = -EPERM;
                     else
                           i = scsi_cmd_ioctl(STp->disk->queue, STp->disk,
                                            file->f_mode, cmd_in, p);
                     if (i != -ENOTTY)
                           return i;
                     break;
       }
\end{lstlisting}
\end{table}

The cleanest way to do it seems to allowed the one we need on the main-binary and the acquired them in the forked process
\section{Pending questions}
\begin{itemize}
\item{}Is the option \verb#ST_BUFFER_WRITES# from \verb#castor.conf# still used?
\item{}Why does\verb#/etc/castor/TPCONFIG# have a tape density column? In order to ject incompatibel tapes from being mounted.
\end{itemize}

