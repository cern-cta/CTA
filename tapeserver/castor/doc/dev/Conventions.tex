%%%%%%%%%%%%%%%%%%%%%%%%%%%%%%%%%%%%%%%%%%%%%%%%%%%%%%%%%%%%%%%%%%%%%%%%%%%%%%
%                      Conventions.tex
%
% This file is part of the Castor project.
% See http://castor.web.cern.ch/castor
%
% Copyright (C) 2003  CERN
% This program is free software; you can redistribute it and/or
% modify it under the terms of the GNU General Public License
% as published by the Free Software Foundation; either version 2
% of the License, or (at your option) any later version.
% This program is distributed in the hope that it will be useful,
% but WITHOUT ANY WARRANTY; without even the implied warranty of
% MERCHANTABILITY or FITNESS FOR A PARTICULAR PURPOSE.  See the
% GNU General Public License for more details.
% You should have received a copy of the GNU General Public License
% along with this program; if not, write to the Free Software
% Foundation, Inc., 59 Temple Place - Suite 330, Boston, MA 02111-1307, USA.
%
%
% Coding conventions and practices chapter of the castor developer
%
% @author Castor Dev team, castor-dev@cern.ch
%%%%%%%%%%%%%%%%%%%%%%%%%%%%%%%%%%%%%%%%%%%%%%%%%%%%%%%%%%%%%%%%%%%%%%%%%%%%%

\chapter{Coding conventions and practices}

\section{Code repositories}

\section{\cpp coding conventions}

\section{Exceptions}

Most of the Object Oriented languages are providing exception handling. This
is a mechanism that allow easy error manipulation and avoids many gotos or
if blocks. The \cpp syntax fro exceptions looks like this~:
\begin{verbatim}
try {
  ...
  castor::exception::Internal error;
  ...
  throw error;
  ...
} catch (castor::exception::Exception e) {
  ...
}
\end{verbatim}
You can throw any object as an exception but in the castor code,
you're supposed to only send objects inheriting from
castor::exception::Exception.
This allows to catch all castor exception in one statement at the
end of the try - catch block.

Catching all exceptions in all methods is not mandatory at all.
However, if some exception may go out of a given method, it should
be mentionned in the method declaration through the \verb1throw1
keyword~:
\begin{verbatim}
  int foo(int param) throw(castor::exception::Exception);
\end{verbatim}
Ideally, all method declarations should have a throw statement, be it
empty. In that case, no exception can come out of the method except for
the standard ones. Note that in such a case, you can still throw a non
standard exception inside the method and the compiler won't complain.
But at run time, your exception will cause \verb1unexpected1 to be
called, leading to exit.

\section{Logging}