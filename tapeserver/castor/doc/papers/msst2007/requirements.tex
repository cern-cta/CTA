We consider CASTOR requirements to fall into three categories:
\begin{itemize}
\item high level requirements
\item functional requirements
\end{itemize}
This section provides a non-exhaustive outline of these main
requirement areas. Further details are available in \cite{castor}.

\subsection{High level requirements}

CASTOR's primary requirement is to provide the data storage capacity
and performance for managing the data produced by the
experiments which use the new CERN accelerator, the LHC. These data
will be processed globally using the LHC Computing Grid\cite{LCG}.

There are three principal levels in this grid. The main data source,
i.e. the ATLAS, ALICE, CMS and LHCb experiments, is named Tier 0. Tier
0 needs to store all data coming from the experiments and to run the
initial data reconstruction. Data are then replicated to Tier 1
centers where further reconstruction and analysis takes place. Finally
many Tier 2 sites are linked to each Tier 1 and act as data customers
for the physics analysis data. 

CASTOR is primarily focused on supporting the Tier 0 and Tier 1 requirements.
Its three main targets are:

\begin{itemize}
\item to provide the Tier 0 Central Data Recording facility (CDR) for the four 
primary LHC experiments and in addition to provide the storage for the 
associated Reconstruction facilities.
\item to handle the massive data transfers to the Tier 1 sites, concurrently 
with new data acquisision.
\item to be used as the storage element of the CERN Tier 1 analysis 
facility that is used by CERN's analysis computing farms.
\end{itemize}

Figure \ref{fig:rates} gives a graphical view of the different concurrent activities
CASTOR is handling.

\begin{figure}[htbp]
\centering
\includegraphics[scale=.17]{rates.eps}
\caption{CASTOR data rates}
\label{fig:rates}
\end{figure}

In addition to the primary LHC experiments CASTOR must also support
the storage requirements of additional, albeit smaller, experiments.

\subsection{Functional Requirements}

Functionality requirements can be summarized in one task: provide
transparent tape storage to the user. In more detail, there are three
core functional requirements for CASTOR:

\begin{itemize}  
\item provide a unique global, hierarchical namespace. This should allow
users to name files in a UNIX like hierarchy and locate them by name afterwards
\item provide transparent tape media management via a disk cache.
Users should not have to consider whether their files are on tape or on
disk
\item provide automated disk cache management. This should ensure
performance and reliability, from both the user and the tape system
point of view.
\end{itemize}

In addition to these core requirements, CASTOR has to interface with
the Grid, the global framework which most physics software
will use to access data. One of the features that characterise a Grid
service, in its broader sense, is the ability to interface with any
underlying fabric resource. The current state of the art in the storage
domain is the Storage Resource Manager interface(SRM) \cite{SRM}.
This interface has to be implemented for CASTOR.

